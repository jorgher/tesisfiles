%% LyX 2.1.3 created this file.  For more info, see http://www.lyx.org/.
%% Do not edit unless you really know what you are doing.
\documentclass[english,spanish]{report}
\usepackage[T1]{fontenc}
\usepackage[utf8]{inputenc}
\usepackage[spanish]{babel}
\usepackage{geometry}
\geometry{verbose,tmargin=3cm,bmargin=3cm,lmargin=4cm,rmargin=3cm}
\setcounter{secnumdepth}{3}
\setcounter{tocdepth}{3}
\usepackage{float}
\usepackage{graphicx}
\usepackage{setspace}
\PassOptionsToPackage{normalem}{ulem}
\usepackage{ulem}
\onehalfspacing


\begin{document}

\chapter{Calidad y Control Estadístico del Proceso (CEP)}


\section{Significado de calidad y mejora de la calidad}

La calidad de un producto o servicio es uno de los factores importantes
en el que los usuarios basan su decisión para la adquisición del mismo,
de ahí que las compañías adopten técnicas y filosofías tendientes
a mejorar la calidad de los productos y servicios que ofrecen.

Es entonces que una compañía debe establecer políticas en relación
a cómo lograr este objetivo, ya que, dependerá del tipo de empresa
la relevancia que le de a la técnica y filosofía empleada, el Dr Garvin
(1987) hace referencia a 8 componentes de la calidad que una organización
debe considerar para establecer su política y programa de calidad.

Desempeño del producto.

Confiabilidad.

Durabilidad.

Facilidad de mantenimiento y servicio.

Estética.

Accesorios.

Calidad percibida (histórica)

Cumplimiento de estándares.

Definir entonces la calidad de un producto no es una tarea simple
y no necesariamente se debe enfocar en un solo criterio; Un enfoque
es que un producto es de calidad si es ``adecuado al uso'', es decir,
que el producto satisface adecuadamente la necesidad por la cuál fue
desarrollado. Pero existen diferentes niveles ya que, un producto
puede satisfacer una necesidad de uso pero a su vez el cliente puede
esperar más en términos de otro criterio de los anteriormente mencionados,
por ejemplo un automóvil en general satisface la necesidad de transporte,
pero existen otros aspectos que distinguen a los diferentes automóviles
(estéticos, potencia, accesorios, durabilidad, facilidad y costos
de mantenimiento, etc) y por tanto se pudiera pensar que éstos tienen
un mayor grado de calidad.

Otro enfoque para definir la calidad ha sido el que los productos
deben ``cumplir las especificaciones'' mismo que dada su concepción
se dirige hacia la verificación de especificaciones en la elaboración
del producto, aspectos relacionados con control de calidad (inspecciones,
pruebas) y aseguramiento de calidad (capacitación, ambiente laboral,
etc). El criterio anterior nos conduce a pensar que la calidad esta
entonces relacionada exclusivamente con la elaboración de los productos
o servicios.

Sin embargo un criterio general pudiera ser que una vez que el diseño
esta definido, es decir que, en la concepción del producto se tomaron
en cuenta todas las característica mencionadas por Galvin. Se deben
establecer los procesos necesarios para llegar a su consumación y
es dentro de estos procesos en donde se presentan diferentes características
críticas que son necesario controlar para garantizar la conformancia
con las especificaciones y lograr que el producto satisfaga la necesidad
para la cuál fue desarrollado.

Pero entonces ¿cuál es el criterio que nos permite lo anterior?, y
es que a través de la historia relacionada con la manufactura de bienes
se ha observado que la ``dispersión'' en las dimensiones de una
especificación propician fallas que afectan directamente el desempeño
del producto.

Por lo anterior se hace necesario \uline{controlar esta variación}
y se llega así a una definición de calidad que es aceptada en general
y es que la \textbf{calidad es inversamente proporcional a la variabilidad}.

Es decir a menor dispersión de los valores con respecto a un valor
nominal los productos desempeñan mejor su función, tienen menos descomposturas
y sobre todo el costo de fabricación se reduce. De ahí la importancia
de una mejora continua enfocada a la \uline{reducción de la variación
de las características críticas del producto}.

Debido a que la variación es un término estadístico, es necesario
para controlar la misma el uso de métodos estadísticos, es común entonces
la clasificación de las características críticas que pueden ser de
dos tipos atributos o variables, la diferencia fundamental es que
las variables son continuas, o bien pueden ser medidas (con algún
instrumento) y los atributos son variables discretas, pueden ser contadas
(número de errores).

Estas características se evalúan con respecto a especificaciones relacionadas
con dimensiones para el caso de variables o brillo de una superficie
cuando hablamos de atributos.

En el diseño las especificaciones se establecen con una tolerancia
y esto permite definir el limite superior e inferior de la especificación,
Tradicionalmente los ingenieros de diseño trabajan sus proyectos sin
considerar otros factores que pueden afectar la calidad, factores
relacionados con la manufactura del producto (materiales, maquinaria,
métodos, medio ambiente, capacitación) de ahí que recientemente se
ha considerado que el proceso de diseño tiene que ser un trabajo de
equipo involucrando a ingenieros de manufactura, calidad, supervisores
de producción, etc, y se ha llamado a este nuevo enfoque \textbf{ingeniería concurrente}.


\section{Una pequeña historia del control y mejora de la calidad}

La calidad es virtualmente una parte integral de los bienes y servicios,
sin embargo métodos formales de mantenimiento y control no se han
desarrollado sino a últimas fechas, el desarrollo de la manufactura
moderna surge con los trabajos de Frederic W. Taylor que enfatiza
en la necesidad de cambiar los sistemas de manufactura con la finalidad
de incrementar la productividad; antes se responsabilizaba a una persona
de la manufactura de un producto (artesano), Taylor establece que
es posible dividir la manufactura de un producto en varios pasos y
al hacer esto, se incrementara la cantidad de artículos producidos.

Posteriormente los trabajos de Frank Gilbreth y otros, se enfocan
en mejorar los métodos y movimientos que se usan en cada uno de los
pasos con la finalidad de establecer tiempos estándares de manufactura.
Todo esto sin considerar que la calidad del producto se verá afectada
por el hecho de que mas personas están involucradas en su fabricación,
además de que esta estandarización conduce a la mecanización y a inhibir
la innovación y la mejora continua.

En 1924 Walter A. Shewhart de Bell Telephonics desarrolla el concepto
de \textbf{control estadístico}, a finales de los años 20 Harold F.
Dodge y Harry G. Roming desarrollan el \textbf{muestreo de aceptación}
como una alternativa a la inspección 100 por ciento.

Es en la segunda guerra donde se implementa el control estadístico
con la finalidad de controlar y mejorar la calidad de los productos.
La ASQC se forma en 1946 con la finalidad de promover el uso de métodos
estadísticos para el control de la calidad. En las décadas de los
50\'{s} y 60\'{s} surge la ingeniería de confiabilidad y el punto
de vista que la calidad es una forma de administrar las organizaciones.

En 1950 se comienza con el uso del diseño de experimentos en los Estados
Unidos. No es sino a partir de la década de los 80\'{s} cuando se
enfatiza el uso de métodos estadísticos para la mejora de la calidad
y la reducción de los costos de manufactura.


\section{Métodos estadísticos para el control de calidad y el mejoramiento}

La base para definir que métodos estadísticos se se debe usar para
la mejora de la calidad de un producto, es el concepto de proceso,
entradas, salidas, actividades que transforman las entradas y un mecanismo
de control; dentro de las entradas se pueden dividir en controlables
y no controlables. 

Para controlar las actividades que transforman las entradas (materias
primas, subesambles, etc) es necesario establecer métodos estadísticos
el más común es el uso de gráficos de control.


\subsection{Gráficos de Control. }

Con su uso, es posible medir el desempeño de las características críticas
(longitud, dureza, etc), lo cuál asegura la calidad del producto.
La finalidad de la gráfica es comparar la variación entre la muestra
tomada en un determinado instante y la variación entre muestras a
lo largo del ciclo de producción.

La comparación se hace con respecto de limites de control establecidos
preliminarmente y si en algún momento del proceso, la toma de datos
y los estadísticos generados (media-rango) se comportan de manera
anormal, se debe detener el mismo y encontrar la causa o causas que
provocan la variación.

Es decir, las gráficas son una herramienta que permite el monitoreo
de las características identificadas como críticas asegurando la calidad
del producto.

La siguiente figura es un ejemplo de gráfica de medias.

\begin{figure}[H]
\centering
\protect\includegraphics[scale=0.5]{\string"tesis/graficacontrolmedia\string".png}
\caption{Proceso en control estadístico}
\label{f1}
\end{figure}


De acuerdo al criterio de normalidad Figura \ref{f2} la dispersión y el
punto medio se mantendrán dentro de ciertos límites y al observar
que los estadísticos (media y desviación estándar) sufren variaciones
anormales es tiempo de revisar el proceso para identificar la causa
o causas que provocan que el comportamiento \textbf{Normal} se halla
modificado.

\begin{figure}[H]
\centering
\protect\includegraphics[clip,scale=0.6]{\string"Home/Dropbox/TESIS/imagenes/tres sigma\string".png}
\caption{\footnotesize{Un comportamiento \textbf{Normal} se refiere a que en general en un proceso, 68.2 \% de los datos se
agrupan a mas menos una desviación estándar, 95.4 \% a mas menos dos
desviaciones estándar y 99.74 \% de los datos a mas menos tres desviaciones
estándar.}}
\label{f2}
\end{figure}


Las causas regularmente se identifican en dos las que son \textbf{inherentes}
(comunes) al proceso que lo hacen variar pero que no afectan la calidad
final del producto, con la presencia de éstas se calculan los límites
de control que permiten monitorear el proceso; eliminar las causas
comunes requiere de un modificación mayor. Es importante mencionar
que los límites de control no corresponden a los límites de especificación
establecidos para la característica estudiada.

Las causas que por lo regular propician que el proceso se salga de
control (Figura 3) se les denomina \textbf{asignables} (especiales)
y son causas que no son parte del mismo sino que se presentan de manera
esporádica, como por ejemplo cuando una herramienta se desgasta, el
descuido de un operador al momento de ajustar un proceso, problemas
con los instrumentos de medición o materia prima, un accidente de
trabajo, entre otros.

\begin{figure}[H]
\centering
\includegraphics[scale=0.7]{\string"Home/Dropbox/TESIS/imagenes/fuera de control\string".jpg}
\caption{Punto fuera de limite inferior}
\label{f3}
\end{figure}

Con lo anterior se puede ver que, las gráficas de control permiten
predecir el comportamiento del proceso dentro de ciertos límites con
los que se establece la hipótesis nula $H_{0}:$ ``el proceso se
encuentra dentro de control estadístico''.

Cuando se presenta un punto fuera de éstos se tiene la posibilidad
de cometer ya sea el error tipo I, error $\alpha$ (Rechazar $H_{0}$
cuando es verdadera, ``riesgo del fabricante''), o tipo II, error
$\beta$ (No rechazar $H_{0}$ cuando es falsa, ``riesgo del consumidor'').


\paragraph{Límites de control.}

Es una práctica común establecer los limites de control a una distancia
de 3$\sigma$ de la media es decir $LSC=\mu+3\sigma$ y $LIC=\mu-3\sigma$
donde $LSC$ significa Límite Superior de Control y $LIC$ Límite
Inferior de Control, en este intervalo se concentraran $99.74\%$
de los valores de la característica distribuidos normalmente.

La especificación de los límites es una decisión importante ya que
influyen directamente en los errores Tipo I y II, mientras mas alejados
de la media se disminuye la probabilidad del error Tipo I pero a su
vez aumenta la probabilidad del error Tipo II. Notese en la Figura \ref{f2} que
el empleo de limites de control 3 sigma implica que $\alpha=0.0027$,
esto es la probabilidad de que la variable esté fuera de los límites
de control cuando el proceso está en control estadístico es $p=0.0027$. 

El tamaño de la muestra es importante para establecer los límites
ya que su relación es inversa; si la muestra es grande, los límites
se encuentran mas cercanos al valor medio, esto permite detectar tendencias
o corrimientos en el los valores de las características que se están
monitoreando en el proceso.

También es importante establecer la frecuencia del muestreo, lo ideal
en todo caso son muestras grandes con pequeños intervalos entre ellas,
esto por lo regular no es posible por cuestiones prácticas y de costo,
entonces si se considera la probabilidad $p$ es factible encontrar
una cantidad probable de observaciones que estarán en control antes
de una fuera, a esta medida se le denomina longitud de corrida promedio
$(ARL)$ de sus iniciales en ingles. 
En esencia, la $ARL$ es el número promedio de puntos que deben graficarse antes de que uno de ellos
indique una condición de proceso fuera de control. 
Para cualquier gráfica
de control de Shewhart la $ARL$ se calcula a partir de una variable
aleatoria geométrica como

\begin{eqnarray*}
ARL & = & \frac{1}{p}\\
 & = & \frac{1}{0.0027}\\
ARL & \cong & 370
\end{eqnarray*}


es decir que si el proceso está bajo control estadístico \footnote{Un proceso esta bajo control estadístico cuando somos capaces de predecir con una probabilidad de error dada, los resultados de su próxima ejecución.} no se presenta una señal fuera de fuera de control por cada 370 observaciones en promedio.


\paragraph{Patrones en gráficas de control}

Una gráfica de control puede indicar una condición fuera de control
cuando se presenta uno o más puntos fuera de los limites Figura \ref{f3},
también cuando la distribución de puntos en un lapso de tiempo presentan
características de anormalidad, es decir, corridas, una serie de puntos
(8 o mas) indicando una tendencia ascendente o descendente, patrones
cíclicos (clusters), después de cierto número de observaciones se
presentan puntos que se repiten Figura \ref{f4}.

\begin{figure}[H]
\centering
\includegraphics[scale=0.5]{\string"Home/Dropbox/TESIS/imagenes/causas fuera control\string".jpg}}
\caption{\textbf{Diferentes condiciones fuera de control}}
\label{f4}
\end{figure}


El problema en la operación de las cartas de control es entonces reconocer
cuando se presentan las condiciones mencionadas, por tanto es necesario
capacitación adecuada. Una guía práctica para la identificación de
patrones puede ser la establecida en el Western Electric Handbook
que dice, Un proceso está fuera de control cuando se presenta:
\begin{itemize}
\item Un punto cae más allá de los límites de control 3-sigma.
\item Dos o tres puntos consecutivos caen más allá de un límite 2-sigma.
\item Cuatro o cinco puntos consecutivos están a una distancia de 1-sigma
o mayor de la linea central.
\item Ocho puntos consecutivos de la gráfica del mismo lado de la linea
central.
\end{itemize}

\subsection{Diseño de experimentos}

Es una herramienta que permite identificar las causas clave que influyen
en la variabilidad de las características críticas del proceso. Es
un enfoque que a partir de la variación controlada de los factores
de entrada se identifican los efectos que se tiene sobre los resultados
del proceso.

Uno de los métodos mas usados es el diseño factorial; El diseño de
experimentos es una de las herramientas fuera de linea mas usadas
en el diseño preliminar de los procesos.

Una vez identificadas la variables que afectan las salidas del proceso,
es necesario modelar estas relaciones de entrada salida, para lo cuál
se utilizan otros métodos estadísticos como es análisis de regresión
y análisis de series de tiempo.

Estos modelos ayudan a determinar la naturaleza y magnitud de los
ajustes requeridos. En muchos procesos una vez entendida la naturaleza
dinámica de las relaciones entrada salida, es factible de manera rutinaria
hacer ajustes al mismo de tal manera que valores futuros de las características
de calidad del producto se mantengan en especificación. A estas actividades
rutinarias de ajuste se les denomina ingeniería de control.


\subsection{Muestreo de aceptación}

Estrechamente relacionado con actividades de inspección y prueba del
producto, la inspección se puede realizar en mas de una etapa del
proceso, usualmente ocurre a la entrada inspección de recibo y a la
salida, es decir antes de que el producto sea enviado al cliente.

Consiste en tomar una muestra representativa de un lote del producto
y determinar la decisión sobre si aceptar o rechazar en base a los
resultados de la inspección. Así como se menciono anteriormente existen
dos tipos de errores (I y II) discutidos previamente y el riesgo lo
determina el valor de AQL (Average Quality Limit) que se haya acordado
con anterioridad.

Los sistemas modernos de calidad usualmente ponen menor énfasis en
la inspección de aceptación y se procura que el control estadístico
del proceso y el diseño de experimentos sean las herramientas de control
adecuadas. 

El objetivo principal de las actividades de la ingeniería de control
es la reducción sistemática de la variación de las características
críticas de los procesos, lo que conduce a mejor desempeño del producto
y por ende una ventaja competitiva para la empresa.


\section{Aspectos administrativos de la mejora de la calidad}


\subsection{Documentación}

El uso de métodos estadísticos de mejora debe estar acompañado de
una política de calidad de la empresa que le de fuerza y permita una
implementación eficiente y continua. Una administración eficiente
de la calidad debe incluir tres actividades básicas: planificación
de la calidad, aseguramiento de la calidad, control y mejora.

Sin un plan estratégico de calidad que incluya productos, procesos,
recursos humanos, infraestructura y sobre todo clientes y sus necesidades,
la operación del negocio esta condenada a sufrir pérdidas debido a
demoras, desperdicios, retornos de producto, retrabajos, mal ambiente
de trabajo, etc. Es en esta etapa en la que la organización con base
en las ocho dimensiones mencionadas anteriormente define misión, objetivos,
metas e indicadores que le permitan enfrentar los retos que la realidad
actual le imponen.


\subsection{Aseguramiento de la calidad}

Son las actividades que se encaminan a garantizar que los productos
y servicios de la organización cumplan de manera satisfactoria las
necesidades de los clientes internos y externos.

Elementos del aseguramiento son la política de calidad, procedimientos,
instrucciones de trabajo y especificaciones. Todos éstos documentos
instruyen al personal en el qué se debe hacer, cómo hacerlo, con qué
herramientas e instrumentos, además establecen planes de mantenimiento
de equipo, planes de capacitación del personal, programas de superación
y crecimiento en la organización, registros del desempeño de las operaciones
donde las características críticas del producto y servicios sean relevantes
para la mejora continua.


\subsection{Control y Mejora de la calidad}

Involucra las actividades relacionadas con el ``control de las características
críticas del producto'', los métodos estadísticos son las herramientas
clave de estas actividades ya que, el control y reducción de la variación
es un factor clave, por lo que el control estadístico y el diseño
de experimentos sean los métodos primordiales en esta etapa de la
operación. 

La mejora continua se hace usualmente proyecto a proyecto e involucra
trabajo en equipo y también el uso de herramientas pensadas para esta
actividad, Diagramas de Pareto, Diagramas de Causa Efecto (Ishikawa),
Metodología de Solución de Problemas, ``Planear, Implementar Controlar
y Mejorar'', son los verbos sustanciales de ésta etapa.

Por lo anterior es posible concluir que la administración de la calidad
es para la organización de nuestro tiempo una necesidad fundamental
y todo administrador que busque el crecimiento de su organización
además de la creatividad que implica el diseño y concepción de productos
y servicios debe fomentar en su personal una filosofía enfocada a
la búsqueda constante de la calidad y mejora.


\section{Filosofías de Calidad y Estrategias Administrativas}

Dentro de los principales filósofos de la calidad se encuentra el
\textbf{\emph{Dr Edward Deming}}, que a partir de sus 14 puntos pretende
que las organizaciones mejoren su productividad y por lo tanto su
permanencia en el mercado. Estos puntos se resumen de la manera siguiente:
\begin{enumerate}
\item Crear constancia en el propósito de mejorar los productos y servicios
que se ofrecen.
\item Adoptar por parte de la administración una filosofía en la que la
calidad en lugar de la cantidad sea la que dirija la operación.
\item No adoptar la inspección para garantizar la calidad de los productos
o servicios.
\item No seleccionar a los proveedores con base en el precio incluir la
calidad de sus productos.
\item La mejora continua de los procesos es un elemento que garantiza la
permanencia del negocio.
\item Capacitación enfocada en la calidad usando métodos que así la garanticen.
\item Liderazgo y supervisión del personal enfocada en la mejora.
\item Eliminar las barreras de comunicación empleado supervisor gerencia.
\item Trabajo en equipo
\item Eliminar metas y eslóganes sin un plan que permita el logro de estas.
\item Eliminar los estándares de producción y metas numéricas impuestas
históricamente sin la consideración de las características de calidad
que deben ser satisfechas en la operación.
\item Es importante que la administración escuche al personal, sugerencias,
comentarios y quejas de los empleados.
\item Instituir un programa de capacitación continuo para todos los empleados.
\item Crear una estructura en la alta gerencia que de seguimiento y refuerce
la implementación de los 13 puntos anteriores.
\end{enumerate}
De lo anterior se puede ver que el Dr. Deming hace un vigoroso énfasis
en el cambio organizacional, pero también se enfoca en los detalles
del cambio indicando que se debe comenzar con la identificación de
los factores que afectan la calidad y a partir de ahí, los métodos
estadísticos tienen su mayor contribución. Control Estadístico del
Proceso y el Diseño de Experimentos.

Deming frecuentemente hablo y escribió acerca de los siete pecados
de la administración.
\begin{enumerate}
\item Falta de constancia en el propósito
\item Énfasis en las utilidades a corto plazo
\item Evaluación del desempeño 
\item Cambios frecuentes de la alta administración
\item Uso exclusivo de estadísticas para orientar el desarrollo de la empresa
\item Costo excesivo de gastos médicos
\item Costos excesivos debidos a penalizaciones legales.
\end{enumerate}
Deming recomienda como guía para la mejora continua el \textbf{ciclo
de Shewhart} que consiste en cuatro pasos: 
\begin{enumerate}
\item \textbf{Planear}. Ante cualquier problema lo primero que debemos hacer
es establecer un programa de trabajo en el que se describan todas
las actividades a realizar con tiempos de entrega y responsables. 
\item \textbf{Hacer}. Ejecutar el plan
\item \textbf{Verificar}. Una vez en la operación llevar registros de los
datos obtenidos.
\item \textbf{Actuar}. Cuando se encuentren variaciones de los resultados
esperados aplicar las herramientas de solución de problemas y establecer
las nuevas actividades de mejora.
\end{enumerate}
Otro gran filosofo y fundador del control de calidad es \textbf{\emph{Joseph
M. Duran }}la administración de calidad de acuerdo a este autor se
debe fundamentar en tres pasos. \emph{Planeación, Control y Mejora}
conocida como la\textbf{\emph{ trilogía de Juran}}, hay que hacer
notar la diferencia con el ciclo de mejora de Shewhart en que estas
fases se refieren al sistema de calidad en su conjunto, de ahí que
cuando se habla de planear se refiere a la identificación de clientes
y sus necesidades para diseñar productos acordes, establecer los procesos
necesarios para realizar los productos y todas las acciones necesarias
para que éstos se desempeñen de acuerdo a lo esperado, Controlar los
procesos sería la siguiente etapa para identificar internamente las
posibles desviaciones de las especificaciones y por último la mejora
continua, identificados los problemas establecer planes y programas
para eliminarlos o en su caso reducir las causas que los provocan.

En la etapa de control es donde los métodos estadísticos mencionados
anteriormente y desarrollados principalmente por Feigenbaum son aplicados. 

Pero \textbf{\emph{Armand V. Feigenbaum}} no solo trabajo sobre los
métodos estadísticos sino que también propuso una filosofía de calidad
llamada \textbf{Control Total de la Calidad} donde propone tres pasos
para mejorar la calidad.
\begin{enumerate}
\item Liderazgo en calidad.\emph{ Planeación, Control y Mejora.}
\item Tecnología de la calidad. 
\item Compromiso de la gerencia.
\end{enumerate}
Así como Deming y Juran el propone su metodología basada en 19 pasos
para la mejora de los procesos, sugiere que gran parte de la capacidad
técnica este concentrada en un área de la organización o un departamento
especializado que contrasta con la visión actual de que el conocimiento
y uso de métodos estadísticos debe ser una actividad que cualquier
departamento tenga las competencias necesarias para llevarlo a cabo.

De lo anterior se puede ver que entre éstos autores existen mas coincidencias
que divergencias en sus filosofías, y que se puede resumir en el hecho
de que todos enfatizan a la calidad como una herramienta que garantiza
la permanencia de la organización, el necesario compromiso de la administración
para la mejora de la calidad y el uso de métodos y técnicas estadísticas
para el logro de ésta transición.

\textbf{\emph{Administración Total de la Calidad (ATC)}} Es una estrategia
para la implementación y administración de la calidad a nivel organizacional,
dando énfasis al cambio cultural, enfoque al cliente, mejora de proveedores,
integración del sistema de calidad con los objetivos de la empresa.

\textbf{\emph{ATC}} ha tenido un éxito moderado por una serie de razones,
entre éstas, falta de compromiso e involucramiento de los altos cuadros
gerenciales, uso inadecuado de herramientas estadísticas, reconocimiento
insuficiente de la importancia en la reducción de la variabilidad
como un objetivo primordial de la empresa, falta de alineación entre
objetivos del sistema de calidad y los objetivos de producción de
la empresa y por ultimo énfasis excesivo en capacitación en lugar
de educación técnica.

Durante el periodo de auge de este programa surge otro desarrollado
por \textbf{\emph{Bill Crosby}} denominado la \textbf{La calidad es
gratis} que busca reducir lo que denomina \textbf{Costos de Calidad}
identificados como:
\begin{itemize}
\item Costos de prevención.

\begin{itemize}
\item Planeación de la calidad e ingeniería
\item Revisión de nuevos productos
\item Diseño de productos y procesos
\item Control de procesos
\item Costos de confiabilidad
\item Entrenamiento y capacitación
\item Registro, Control y Análisis de información
\end{itemize}
\item Costos de evaluación (detección)

\begin{itemize}
\item Inspección y prueba de recibo de materiales
\item Inspección y prueba de productos (en proceso y final)
\item Materiales y servicios consumidos
\item Calibración del equipo de prueba
\end{itemize}
\item Costos de falla interna

\begin{itemize}
\item Desperdicios
\item Retrabajo
\item Análisis de fallas
\item Tiempo muerto
\item Pérdidas de productividad
\end{itemize}
\item Costos de falla externa

\begin{itemize}
\item Ajustes por quejas de clientes
\item Productos regresados
\item Costos de seguros y fianzas
\item Costos por garantías
\item Costos indirectos
\end{itemize}
\end{itemize}
Este programa clama que por cada dolar invertido en Reducir Costos
de Prevención redundará en un ahorro de 10 dolares en Costos de Falla
Interna y Externa; de ahí la importancia de llevar el control y registro
de éstos.

Para obtener reducciones en los costos se hace necesario entonces
una metodología que incluya la detección, análisis y reducción de
los costos de calidad, herramientas útiles para analizar son los diagramas
de \textbf{Pareto}

\begin{figure}[H]
\centering
\includegraphics[scale=0.5]{\string"Home/Dropbox/TESIS/imagenes/paretofallainterna\string".png}
\caption{\textbf{Costos de Calidad}}
\label{f5}
\end{figure}
 En la figura podemos observar que la falla interna es el elemento
relevante de costos, pero además podemos desglosar e identificar aquellos
factores dentro de la categoría que aportan mayormente. ver siguiente
gráfica.

\pagebreak
\begin{figure}[H]
\centering
\includegraphics[scale=0.5]{\string"Home/Dropbox/TESIS/imagenes/paretodesglosefallainterna\string".png}}
\caption{\textbf{Desgloce falla interna}}
\label{f6}
\end{figure}

 llegando asía a identificar factores relevantes para utilizar herramientas
como el diagrama de Ishikawa\textbf{}\footnote{\textbf{El Dr. Kaoru Ishikawa. }Las principales ideas de Ishikawa
se encuentran en su libro ¿Qué es el control total de calidad?: la
modalidad japonesa. Donde hace énfasis en las diferencias culturales
entre Occidente y Japón para la implementación y administración del
\textbf{\emph{Control Total de Calidad (CTC)}}}\textbf{ }que permite a través de un desglose con las 5 m's \emph{\uline{Método,
Material, Medio Ambiente, Mano de Obra y Mediciones}} llegar a determinar
que causas son las que provocan los problemas identificados.



\pagebreak
\begin{figure}[H]
\centering
\includegraphics{Home/Dropbox/TESIS/imagenes/diagrama_ishikawa.jpeg}}
\caption{\textbf{Diagrama ISHIKAWA}}
\label{f7}
\end{figure}


Pero bien ¿Que tanto es posible reducir los costos de calidad?.

Si bien en algunas organizaciones es posible reducir los costos de
calidad significativamente, el llegar a cero costos no es realista,
debido a que para llegar a cero sería necesario incrementar los costos
de prevencion, por lo que el adecuado balance que beneficie a la organización
es también un elemento a considerar cuando se adopta esta metodología.

Otro programa iniciado por la empresa \textbf{Motorola} se ha convertido
también en una filosofía que algunas organizaciones han tomado, este
programa se llama \textbf{\emph{Seis Sigma}} toma su nombre de la
distribución normal que como se mencionó anteriormente nos permite
identificar la variación de las características del producto que afectan
significativamente a la calidad.

El concepto seis sigma de Motorola consiste en reducir la variación
en el proceso de tal manera que los límites de especificación se encuentren
a seis veces la desviación estándar del valor medio (especificación).
Lo que conduce a tener únicamente 2 partes por billón producidas.

\pagebreak
\begin{figure}[H]
\centering
\includegraphics[scale=0.5]{Home/Dropbox/TESIS/imagenes/MotorolaSeisSigma}}
\caption{\textbf{Efecto del desplazamiento de la media}}
\label{f8}
\end{figure}


Lo anterior se puede verificar en la figura anterior ya que esta presenta
la comparación de una característica de calidad en la cuál se establecen
los límites de especificación a tres desviaciones estándar a cada
lado de la media (a) con un modelo a seis desviaciones estándar (b).
Se puede ver que considerando mas menos seis veces la desviación estándar
en la figura (a) se tiene un porcentaje dentro de especificaciones
de 99.73 que corresponde a 2700 partes por millón (ppm) defectuosas,
pero cuando se analiza que un producto tiene mas de un componente
la probabilidad de error se incrementa potencialmente es decir $p^{n}$
por ejemplo si el producto final constara de 100 componentes la probabilidad
de error final es $0.9973^{100}=0.7631$ es decir 23.7 \% de los productos
estarían por debajo de tres sigma.

De ahí que el concepto seis sigma implica que al colocar los limites
de especificación a seis veces la desviación estándar de la media
la $p=(0.999999998)^{100}=0.9999998$ es decir 2 ppm defectuoso situación
que resulta mucho mayor satisfactoria desde el punto de vista del
usuario final.

En el caso anterior se supone que en la fabricación de los componentes
todas las características de calidad se mantuviera centrada con respecto
a la media (especificación) pero en la practica es común que existan
desviaciones, bajo el enfoque seis sigma se pueden presentar desviaciones
hasta de 1.5 veces la desviación estándar. Bajo este escenario el
porcentaje de piezas dentro de especificaciones sería de 99.999660
(ver Figura 8) lo que nos da un total de piezas aceptables $0.9999966^{100}=0.9996600$
equivalente a 3.4 ppm defectuoso. 

Lo anterior se establece con ciertas reservas ya que, para poder hacer
predicciones con respecto al desempeño del proceso, los valores de
la media y desviación estándar deben mantenerse \textbf{constantes}
o sea, deben estar al rededor del valor medio esperado, lo que se
ha denominado, el proceso debe ser \textbf{estable.}

Si por alguna razón la media se desvía mas de 1.5 veces o la desviación
estándar se incrementa, la predicción de 3.4 ppm no se puede garantizar,
situación que es factible que ocurra en la práctica.

De ahí que para garantizar la estabilidad de los procesos y por ende
la reducción de los costos, una compañía que desea involucrarse en
este programa debe hacer cambios a su estructura administrativa y
a sus políticas de operación en las que se haga énfasis en la importancia
de la filosofía para mejorar procesos de manufactura y de servicios
internos y externos.

Dentro de los cambios en la estructura organizacional es el de incluir
un área específica que administre los esfuerzos en la implementación
del programa contando con un \textbf{equipo líder} encargado de la
supervisión, selección y administración de los proyectos de mejora
propuestos por los \textbf{equipos seis sigma}, éste equipo está integrado
por personal staff y un responsable técnico que se le ha llamado el
maestro cinta negra (Master Black Belts). 

El MBB es un individuo a tiempo completo desarrollando actividades
de identificación, selección y revisión de proyectos, además lleva
a cabo la capacitación a cintas negras (Black Belts) y cintas verdes
(Green Belts).

Cada equipo de mejora cuenta con un \textbf{campeón} cuya responsabilidad
es evaluar y supervisar las actividades del equipo, solicitar y asignar
recursos técnicos y materiales, así como eliminar obstáculos que impidan
el desempeño exitoso del mismo. 

El rol de campeón no es de tiempo completo y por lo regular tiene
asignado mas de un proyecto, en los equipos hay un líder que puede
ser un cinta negra (Black Belt) asignado tiempo completo que se asegura
de que el proyecto se desarrolle en los tiempos y con los recursos
asignados, los miembros de los equipos se escogen de diferentes áreas
de la organización y son seleccionados con base en su experiencia,
por lo regular dedican un 25 \% de su tiempo a las actividades del
equipo. 

Los Green Belts (cintas verdes) cuyo conocimiento de las herramientas
seis sigma es menor que el de los cintas negras, por lo regular solo
son miembros de los equipos seis sigma.

La siguiente figura muestra la estructura típica de la organización
seis sigma.

\pagebreak
\begin{figure}[H]
\centering
\includegraphics[scale=0.6]{\string"Home/Dropbox/TESIS/imagenes/Estructura Equipo Seis Sigma\string".png}}
\caption{\textbf{Organización equipos seis sigma}}
\label{f9}
\end{figure}


Los proyectos en los que los equipos regularmente se ven involucrados
se relacionan con:
\begin{itemize}
\item Reducción de los ciclos de manufactura.
\item Eliminación de retrabajos y errores en los procesos
\item Reducción de faltantes en inventarios 
\item Reducción de variaciones en procesos no relacionados con la manufactura
de productos (financieros, ventas, contables, etc)

\begin{itemize}
\item Reducción de errores en la nomina
\item Reducción de errores en las ordenes de compra
\item Reducción de errores en los estados financieros
\item Minimizar costos de consultorías, etc.
\end{itemize}
\end{itemize}

\paragraph{Metodología de mejora continua (DMAIC)}

Un procedimiento de solución de problemas estructurado usado en los
equipos seis sigma es el que se conoce como DMAIC donde las siglas
quieren decir:
\begin{itemize}
\item Define (Definir) 
\item Measure (Medir)
\item Analize (Analizar)
\item Improve (Mejorar)
\item Control (Controlar)
\end{itemize}

\subparagraph{Define.}

En esta etapa los equipos identifican las oportunidades de mejora,
definen los objetivos de los clientes, mapean los procesos que permiten
satisfacer tales objetivos y establecen un plan a seguir.


\subparagraph{Measure}

Definen que características medir, se establecen las actividades de
recolección de datos, desarrollan las herramientas o controles si
no existen y/o validar las existentes, establecen los niveles sigma
de desempeño.


\subparagraph{Analize}

Se analizan los datos para encontrar causas potenciales de variación
de los objetivos identificados, Se calcula la capacidad de los procesos,
el flujo de los procesos y su tiempo de ciclo.


\subparagraph{Improve}

Generar y cuantificar soluciones posibles, evaluar y seleccionar la
solución final, verificar y sustentar el proyecto para la aprobación
final.


\subparagraph{Control}

Desarrollar planes de monitoreo y control, desarrollar procesos a
prueba de fallas, controlar las características críticas definidas

Entre cada actividad los equipos tienen reuniones de presentación
de avances y resultados a los gerentes y dueños del proceso, en estas
reuniones se analizan los avances para ver si se va a terminar el
proyecto en el tiempo planeado o hacer ajustes al mismo, se discuten
posibles modificaciones al proyecto con base en aportaciones de los
dueños y gerentes, además se discuten e identifican posibles barreras
que estén afectando el desempeño del equipo y se establezcan las estrategias
para solventarlas o eliminarlas.

Estas reuniones de evaluación son un paso determinante para el buen
funcionamiento del programa seis sigma y es de vital importancia que
estas se realicen inmediatamente después de que el equipo termine
cada etapa.

La estructura DMAIC favorece el pensamiento creativo y facilita los
cambios al proceso dentro del ámbito del mismo, en caso de que el
proceso no sea capaz de realizar los objetivos identificados en la
etapa uno, se hace necesario un rediseño del mismo, La metodología
\textbf{Seis Sigma} tiene contemplado otra metodoligía se le denomina
diseño para seis sigma (DFSS).

Una de las razones por las que se ha identificado que DMAIC es tan
efectivo es que hace uso de un conjunto de herramientas que en su
conjunto favorecen el proceso de solución, estas herramientas son:
\begin{enumerate}
\item Planeación y Calendarización de Proyectos. (Gantt Charts)
\item Diagramas de Flujo (Mapeo) de Procesos.
\item Análisis de Causa-Efecto. (ISHIKAWA)
\item Análisis de Capacidad de Proceso. (Cpk, Ppk)
\item Prueba de Hipótesis e Intervalos de Confianza.
\item Análisis de Repetibilidad y Reproducibilidad de métodos de medición.
\item Análisis de Modo y Falla Potencial.
\item Control Estadístico del Proceso.
\end{enumerate}
Es importante resaltar aquí para fines del presente trabajo, que todos
éstos métodos se enfocan en el control de una variable en particular
(temperatura, espesor, altura, diámetro, cuentas por cobrar, tiempo
de entrega, inventario de un artículo, etc) a este tipo de métodos
los denominamos \textbf{univariados.}


\subparagraph{Selección de proyectos.}

Para la selección de un proyecto es importante identificar los beneficios
económicos así como el \textbf{cambio sustancial} (breakthrough) a
la operación de la organización, el \textbf{valor de oportunidad}
de los proyectos debe ser un elemento de selección importante y es
conveniente que la organización implemente un sistema adecuado para
la medición del valor agregado de los proyectos, es decir que, estén
alineados a los objetivos, reduzcan costos, eliminen errores, reduzcan
el tiempo de ciclo del proceso, el tiempo de entrega a clientes externos
o internos, mejoren las condiciones de trabajo (seguridad), uso eficiente
de los recursos, además se busca que se mejore la capacidad de diseño
de proceso, introducción de nuevos productos y servicios al mercado
así como, mejora en los procesos de mercadotecnia y venta, en fin
toda una gama de posibilidades de mejora.

Es por eso que una organización enfocada en \textbf{Seis Sigma} mejora
sus índices de productividad y por ende los indicadores financieros
haciéndola mas interesante por los dividendos que pagará a los inversionistas,
retorno de inversión y ganancia por acciones.

Por tanto, la selección de proyectos es una de las actividades preponderante
del programa, principalmente los proyectos deben tener una duración
adecuada y aportar a los indicadores relevantes de la organización
para ser aprobados. 

Es por lo regular como método de seguimiento a proyectos un \textbf{panel
de control} en el que se puedan verificar los avances de los proyectos
y sus impactos en los indicadores organizacionales.

Un ejemplo de panel de control se muestra en la siguiente gráfica.

\pagebrea
\begin{figure}[H]
\centering
\includegraphics[scale=0.6]{\string"Home/Dropbox/TESIS/imagenes/lean six sigma bsc\string".jpg}}
\caption{\textbf{Panel de control}}
\label{f10}
\end{figure}



\section{Control estadístico de procesos multivariados}

Como se ha podido observar a lo largo de la discusión anterior, la
adopción de una filosofía de mejora es un elemento importante para
la mejora de la calidad de sus productos y servicios. Y al hacerlo
también se resalto que las herramientas estadísticas son una parte
importante de los componentes de la filosofía, por tanto contar con
los recursos humanos y técnicos sensibilizados y capacitados es un
factor relevante para el éxito.

Es también de mencionar importante que el compromiso de la gerencia
para la adopción de la filosofía es preponderante, ya que si éste
cualquier programa que se pretenda implantar resultara en tiempo y
dinero malgastado.

Usualmente por facilidad en el entrenamiento e implantación los métodos
\textbf{univariados}, es decir aquellos que se enfocan en una variable
son los que han tomado el papel principal en su uso, pero en la practica
es normal encontrar que en un proceso existe mas de una característica
a controlar y que además éstas se relacionan unas con otras. 

De ahí que incorporar conceptos de estadística \textbf{multivariada}
pasa a ser un tema importante en la implementación. En el presente
trabajo se presentarán de manera general herramientas univariadas
y multivariadas, haciendo énfasis en las últimas resaltando su eficiencia
y eliminando algunos mitos por los que se ha evitado su uso
\end{document}
